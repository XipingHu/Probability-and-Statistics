
%%% Local Variables:
%%% mode: latex
%%% TeX-master: "main"
%%% End:

\section{数字特征}

\subsection{数学期望与方差}

\subsubsection{离散型}


\begin{equation*}
  EX=\sum\limits_{i=1}^{\infty} x_{i} p_{i}
\end{equation*}
\begin{equation*}
  DX=E(X-EX)^{2}=EX^{2}-E(X^{2})
\end{equation*}

\subsubsection{连续型}
\begin{equation*}
  EX=\int_{-\infty}^{+\infty} x f(x) \mathrm{d} x
\end{equation*}
\begin{equation*}
  DX=\int_{-\infty}^{+\infty} (x-EX)^{2} f(x) \mathrm{d} x
\end{equation*}



\subsection{不同分布的数学期望与方差}

\subsubsection{离散型}


\begin{table}[H]
  \renewcommand\arraystretch{1.5}
  \begin{tabular}{|l|l|l|}
    \hline
    $X$&$EX$&$DX$\\
    \hline
    $B(n,p)$&$np$&$np(1-p)$\\
    \hline
    $P(\lambda)$&$\lambda$&$\lambda$\\
    \hline
  \end{tabular}
\end{table}

\subsubsection{连续型}

\begin{table}[H]
  \renewcommand\arraystretch{2}
  \begin{tabular}{|l|l|l|}
    \hline
    $X$&$EX$&$DX$\\
    \hline
    $N(\mu,\sigma^{2})$&$\mu$&$\sigma^{2}$\\
    \hline
    $E(\lambda)$&$\dfrac{1}{\lambda}$&$\dfrac{1}{\lambda^{2}}$\\
    \hline
    $U(a,b)$&$\dfrac{b-a}{2}$&$\dfrac{(b-a)^{2}}{12}$\\
    \hline
  \end{tabular}
\end{table}

\subsection{切比雪夫不等式}

\begin{equation}
  \label{eq:9519029192103210}
  P \left( |X-EX| \geq \varepsilon \right) \leq \dfrac{DX}{\varepsilon^{2}}
\end{equation}

\subsection{协方差}

\begin{equation*}
  Cov(X,Y)=E \left[ \left( X-EX \right) \left( Y-EY \right) \right] = E \left( XY \right) -EXEY
\end{equation*}
协方差的性质有
\begin{equation*}
  Cov(aX,bY)=ab Cov(X,Y)
\end{equation*}
\begin{equation*}
  Cov(X_{1}+X_{2} ,Y) = Cov(X_{1},Y) + Cov(X_{2},Y)
\end{equation*}
特别地
\begin{equation*}
  DX=Cov(X,X)
\end{equation*}

\subsection{相关系数}

\begin{equation*}
  \rho_{xy} = \dfrac{Cov(X,Y)}{\sqrt{DX} \sqrt{DY}}
\end{equation*}

\subsection{矩}

\paragraph{$k$阶原点矩}
$\nu_{k}=EX^{k}$

\paragraph{$k$阶原点绝对矩}
$\alpha_{k}=E|X|^{k}$

\paragraph{$k$阶中心矩}
$\mu_{k}=E(X-EX)^{k}$

\paragraph{$k$阶原点绝对矩}
$\beta_{k}=E|X-EX|^{k}$

