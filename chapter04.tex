
%%% Local Variables:
%%% mode: latex
%%% TeX-master: "main"
%%% End:
\section{参数估计}

\subsection{矩估计}

\begin{itemize}
\item 求$EX$与$EX^{2}$
\item $A_{1} = \dfrac{1}{n} \sum\limits_{i=1}^{n} X_{i} = \bar{X}$
\item $A_{2} = \dfrac{1}{n} \sum\limits_{i=1}^{n} X_{i}^{2}$
\item $A_{2} - A_{1}^{2} = \dfrac{1}{n} \sum\limits_{i=1}^{n} X_{i}^{2} - \bar{X}^{2} = \dfrac{1}{n} \sum\limits_{i=1}^{n} \left( X_{i}-\bar{X} \right)^{2} = \tilde{S}^{2}$
\item 用上面的式子解出分布中的参数,消去$A_{1}$和$A_{2}$
\end{itemize}
例子详见p101

\subsection{极大似然估计}

\begin{itemize}
\item 根据概率密度用连乘的形式写出似然方程
\item 对似然方程取对数,求导判断参数取什么值的时候,似然函数最大
\item 如果参数取多个值的时候似然函数都能最大,则取最大值作为极大似然估计
  
\end{itemize}
详见p103

\subsection{无偏估计}

$S^{2}=\dfrac{1}{n-1} \sum\limits_{i=1}^{n} \left( X_{i} - \bar{X} \right)^{2}$是$\sigma^{2}$的无偏估计,$\tilde{S}^{2}=\dfrac{1}{n} \sum\limits_{i=1}^{n} \left( X_{i} - \bar{X} \right)^{2}$是$\sigma^{2}$的有偏估计

\subsection{区间估计(置信水平$1-\alpha$)}

\subsubsection{均值$\mu$的区间估计}


\paragraph{已知$\sigma^2$,由章节\ref{sec:790103959290},推论1:}

$\left( \bar{X} - \dfrac{\sigma}{\sqrt{n}} u_{\alpha/2}, \bar{X} + \dfrac{\sigma}{\sqrt{n}} u_{\alpha/2} \right)$

\paragraph{未知$\sigma^2$,由章节\ref{sec:790103959290},推论2:}

$\left( \bar{X} - \dfrac{S}{\sqrt{n}} u_{\alpha/2}, \bar{X} + \dfrac{S}{\sqrt{n}} u_{\alpha/2} \right)$

\subsubsection{方差$\sigma$的区间估计}

\paragraph{未知$\mu$,由章节\ref{sec:790103959290},方差满足的分布:}

$\left( \dfrac{\sqrt{n-1}S}{\sqrt{\chi^{2}_{\alpha/2} (n-1)}},\dfrac{\sqrt{n-1}S}{\sqrt{\chi^{2}_{1-\alpha/2} (n-1)}} \right)$

\subsubsection{单侧置信区间}

\paragraph{单侧置信下限$\underline{\theta}$}

$P(\theta > \underline{\theta}) = 1 - \alpha$

\paragraph{单侧置信上限$\overline{\theta}$}

$P(\theta < \overline{\theta}) = 1 - \alpha$
