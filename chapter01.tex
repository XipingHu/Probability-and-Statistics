%%% Local Variables:
%%% mode: latex
%%% TeX-master: "main"
%%% End:

\section{随机事件与概率}

\subsection{条件概率}

有事件$A$、$B$,$P(B) \neq 0$,则$A$在$B$发生条件下的概率
\begin{equation*}
  P(A|B) = \frac{P(AB)}{P(B)}
\end{equation*}
事件$B$是发生的前提条件,也有
\begin{equation*}
  \label{eq:40011034953419}
  P(AB) = P(A|B) P(B) = P(B|A)P(A)
\end{equation*}
全概率公式(p9)
\begin{equation}
  \label{eq:1}
  P(B) = \sum_{i=1}^n P(B|A_i) P(A_i)
\end{equation}
贝叶斯公式
\begin{equation*}
  P(A_i | B) P(B) = P(B | A_i) P(A_i)
\end{equation*}
将公式\ref{eq:1}代入
\begin{equation*}
   P(A_i | B) \sum_{i=1}^n P(B|A_i) P(A_i) = P(B | A_i) P(A_i)
 \end{equation*}
最终得到
 \begin{equation*}
    P(A_i | B) = \frac{P(B | A_i) P(A_i)}{ \sum\limits_{i=1}^n P(B|A_i) P(A_i) }
  \end{equation*}
\subsection{随机变量的分布}
\subsubsection{离散型}  

\paragraph{一维分布列}


\begin{table}[H]
\begin{tabular}{c|ccccc}
  $X$ & $x_1$ &$x_2$ & $\cdots$&$x_i$ & $\cdots $\\
  \hline
  $P$&$p_1$&$p_2$&$\cdots$&$p_i$&$\cdots$
\end{tabular}
\end{table}

其中$\sum\limits_i p_i = 1$,分布函数$F(x) = P(X \leq x)$


\paragraph{二维分布列}
\begin{table}[H]
\begin{tabular}{c|ccc|c}
\diagbox{X}{Y} & $y_{1}$ & $y_{2}$ &$y_{3}$& \\
\hline
$x_{1}$ & $p_{11}$ & $p_{12}$ & $p_{13}$& $p_{1\cdot}$\\
$x_{2}$ & $p_{21}$ & $p_{22}$ & $p_{23}$ &$p_{2\cdot}$\\
  \hline
  &$p_{\cdot 1}$&$p_{\cdot 2}$&$p_{\cdot 3}$&
\end{tabular}
\end{table}
其中$\sum\limits_{i,j} p_{ij} = 1$


\paragraph{二项分布$X\sim B(n,p)$}

共实验$n$次,成功的概率是$p$,分布列为
\begin{equation*}
  P(X=k)=C_n^k p^k \left( 1-p \right)^{n-k}
\end{equation*}
分布列先增后减,中心项$P(X=\left[ \left( n+1 \right)p \right])$最大

\paragraph{泊松分布$X \sim P(\lambda)$}

当二项分布的$n$很大,$p$很小时,可近似为泊松分布
\begin{equation*}
  P(X=k)=\dfrac{\lambda^{k}e^{-\lambda}}{k!}
\end{equation*}
先增后减,中心项为$P(X=\left[ \lambda \right])$

\paragraph{超几何分布(略)p19}

\paragraph{几何分布}

分布列为
\begin{equation*}
  P(X=k)=pq^{k-1}
\end{equation*}
计算分布函数要用到等比数列的求和公式

\subsubsection{连续型随机变量}

\paragraph{一维随机变量概率密度与分布函数}

\begin{equation*}
  F(x)=\int_{-\infty}^{x} f(t) \mathrm{d} t = P(X<x)
\end{equation*}

\begin{equation*}
  F(+\infty)=\int_{-\infty}^{+\infty} f(x) \mathrm{d} t =1
\end{equation*}

\paragraph{二维随机变量概率密度与分布函数}
\begin{equation*}
  F(x,y)=\int_{-\infty}^{x} \int_{-\infty}^{y} f(u,v) \mathrm{d} u \mathrm{d}v
\end{equation*}
\begin{equation*}
  F(+\infty,+\infty)=\int_{-\infty}^{+\infty} \int_{-\infty}^{+\infty} f(u,v) \mathrm{d} u \mathrm{d}v=1
\end{equation*}

边缘概率密度
\begin{equation*}
  f_{X}(x)=\int_{-\infty}^{+\infty}f \left( x,y \right) \mathrm{d}y
\end{equation*}
\begin{equation*}
  f_{Y}(y)=\int_{-\infty}^{+\infty}f \left( x,y \right) \mathrm{d}x
\end{equation*}
二维随机变量求分布函数需要画图分区域讨论,详见:p39例3.4
\paragraph{均匀分布$X \sim U(a,b)$}

\begin{equation*}
  f(x)=\left\{
    \begin{array}{cl}
      \dfrac{1}{b-a} &, a\leq x \leq b \\
      0 &, \text{其他}
    \end{array}
    \right.
\end{equation*}

\begin{equation*}
  F(x)=\left\{
    \begin{array}{cl}
        0&,x < a \\
        \dfrac{x-a}{b-a}&, a \leq x < b \\
        1&, x \geq b
    \end{array}\right.
\end{equation*}

\paragraph{指数分布 $X \sim E(\lambda)$}
\begin{equation*}
    f(x)=\left\{
    \begin{array}{cl}
      \lambda e^{-\lambda x} &, x \geq 0 \\
      0 &, x < 0
    \end{array}
    \right.
\end{equation*}
\begin{equation*}
  F(x)=\left\{
    \begin{array}{cl}
        0&,x < 0 \\
        1-e^{-\lambda x}&, x\geq 0
    \end{array}\right.
\end{equation*}

\paragraph{正态分布 $N\sim (\mu,\sigma^2 )$}
\textbf{一维情形}

\begin{equation*}
  f(x)=\dfrac{1}{\sqrt{2\pi}\sigma}e^{-\dfrac{(x-\mu)^{2}}{2\sigma^{2}}}
\end{equation*}
\begin{equation}
  \label{eq:13905171973}
  F(x)=\dfrac{1}{\sqrt{2\pi}\sigma}\int_{-\infty}^{x} e^{-\dfrac{(x-\mu)^{2}}{2\sigma^{2}}} \mathrm{d}x = \dfrac{1}{\sqrt{2\pi}}\int_{-\infty}^{x} e^{-\dfrac{\left(\dfrac{x-\mu}{\sigma}\right)^{2}}{2}} \mathrm{d}\left( \dfrac{x-\mu}{\sigma} \right)
\end{equation}
定义标准正态分布$N\sim (0,1)$的分布函数为
\begin{equation*}
  \Phi(x)=\dfrac{1}{\sqrt{2\pi}} \int_{-\infty}^{x}e^{-\dfrac{t^{2}}{2}} \mathrm{d} t
\end{equation*}
则方程\ref{eq:13905171973}可化为
\begin{equation*}
  F(x)=\Phi \left( \dfrac{x-\mu}{\sigma} \right) = P(X<x)
\end{equation*}
\textbf{二维情形}\\
相关系数为零,则
\begin{equation*}
  f(x)=\dfrac{1}{\sqrt{2\pi}\sigma_{1}}e^{-\dfrac{(x-\mu_{1})^{2}}{2\sigma_{1}^{2}}} \dfrac{1}{\sqrt{2\pi}\sigma_{2}}e^{-\dfrac{(x-\mu_{2})^{2}}{2\sigma_{2}^{2}}}
\end{equation*}
详见p41


\subsubsection{随机变量函数的分布 p28}
对于离散型随机变量,根据分布列求概率就可以。对于连续型,先根据概率求分布函数,再对分布函数求导,得到概率密度。

\paragraph{条件分布}

根据公式\ref{eq:40011034953419},相似的,定义
\begin{equation*}
  f(y|x)=\dfrac{f(x,y)}{f_{X}(x)}
\end{equation*}

为 $X=x$时$Y$的条件密度

\subsubsection{随机变量的独立性}

\paragraph{离散型}
\begin{equation*}
  P(X_{1}=x_{1},X_{2}=x_{2})=P(X_{1}=x_{1})P(X_{2}=x_{2})
\end{equation*}
\paragraph{连续型}
\begin{equation*}
  f(x,y)=f_{X}(x)f_{Y}(y)
\end{equation*}

